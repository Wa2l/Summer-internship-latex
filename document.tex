%----------------------------------------------------------------------------------------
%	PACKAGES AND OTHER DOCUMENT CONFIGURATIONS
%----------------------------------------------------------------------------------------

\documentclass[11pt,fleqn]{book} % Default font size and left-justified equations

\usepackage[top=3cm,bottom=3cm,left=3.2cm,right=3.2cm,headsep=10pt,a4paper]{geometry} % Page margins

\usepackage{xcolor} % Required for specifying colors by name
\definecolor{ocre}{RGB}{243,102,25} % Define the orange color used for highlighting throughout the book

% Font Settings
\usepackage{avant} % Use the Avantgarde font for headings
%\usepackage{times} % Use the Times font for headings
\usepackage{mathptmx} % Use the Adobe Times Roman as the default text font together with math symbols from the Sym­bol, Chancery and Com­puter Modern fonts

\usepackage{microtype} % Slightly tweak font spacing for aesthetics
\usepackage[utf8]{inputenc} % Required for including letters with accents
\usepackage[T1]{fontenc} % Use 8-bit encoding that has 256 glyphs

% Bibliography
\usepackage[style=alphabetic,sorting=nyt,sortcites=true,autopunct=true,babel=hyphen,hyperref=true,abbreviate=false,backref=true,backend=biber]{biblatex}
\addbibresource{bibliography.bib} % BibTeX bibliography file
\defbibheading{bibempty}{}

% Index
\usepackage{calc} % For simpler calculation - used for spacing the index letter headings correctly
\usepackage{makeidx} % Required to make an index
\makeindex % Tells LaTeX to create the files required for indexing
\usepackage{verbatim}

%----------------------------------------------------------------------------------------

\begin{document}
	%----------------------------------------------------------------------------------------
	%	TITLE PAGE
	%----------------------------------------------------------------------------------------
	
	\begingroup
	\thispagestyle{empty}
	%\AddToShipoutPicture*{\put(6,5){\includegraphics[scale=1]{portal-energy-background}}} % Image background
	
	\centering
	\vspace*{9cm}
	\par\normalfont\fontsize{35}{35}\sffamily\selectfont
	Title\par % Book title
	\vspace*{1cm}
	{\Huge Wael ARFAOUI}\par % Author name
	\endgroup
	
	%----------------------------------------------------------------------------------------
	%	Thanks
	%----------------------------------------------------------------------------------------
	\frontmatter
	\addcontentsline{toc}{chapter}{Remerciements}
	
	\begin{center}
		\vspace*{5cm}
		\par\normalfont\fontsize{20}{20}\sffamily\selectfont
		Remerciements
		\par\normalfont\fontsize{14}{14}\sffamily\selectfont
		\vspace*{1cm}
		C’est avec plaisir que je réserve cette page en signe de gratitude et de
		profonde reconnaissance à tous ceux qui m'ont aidé à réaliser ce projet.Je
		tiens à remercier le jury pour l’honneur qu’il m'a fait pour avoir accepté
		de juger mon travail.
		J'exprimes ma gratitude tout particulièrement à M.Sami Bessaies pour son encadrement, son aide, sa disponibilité, ses conseils fructueux et ses encouragements qu’il m'a prodigué tout au long de ce projet.
		Finalement, et avec beaucoup de respect, je ne peux laisser cette occasion
		sans saluer chaleureusement tous mes enseignants de l’ENIT ainsi que
		mes collègues pour leur soutien moral.\par
	\end{center}
	
	\newpage
	\addcontentsline{toc}{chapter}{Résumé}
	\begin{center}
		\vspace*{5cm}
		\normalfont
		{\LARGE Résumé}
		\vspace*{1cm}
	\end{center}
	\normalfont

	Le présent rapport décrit les différents taches effectuées tout au long de mon expérience comme stagiaire chez Poulina Group Holding. Durant ce stage de trois mois, j'ai travaillé sur l'application Bware, application développée par Smartdata technology.\\
	Cette application est un moyen de prévention contre les différents problèmes de santé. L'utilisateur peut surveiller tous les risques qui menacent sa santé. Que ce soit la Pollution, le Pollen, la Grippe, la Varicelle ou la Gastro-entérite, Bware" l'informe continuellement du niveau du risque.\\
	
	Le travail demandé consistait à améliorer l'application en rajoutant des fonctionnalités pour les utilisateurs publiques et d'autres pour le suivi et le contrôle de l'application.

		
	
	\newpage
	\addcontentsline{toc}{chapter}{Glossaire}

	
	%----------------------------------------------------------------------------------------
	%	table of contents
	%----------------------------------------------------------------------------------------
	
	\renewcommand\contentsname{Table des Matières}
	\tableofcontents
	
	%----------------------------------------------------------------------------------------
	%	chapters
	%----------------------------------------------------------------------------------------
	
	
	
\end{document}