\addcontentsline{toc}{chapter}{Introduction}
\chapter*{Introduction}
\vspace*{1cm}
\qquad Le monde entier vie sous le rythme d'une évolution et de croissance rapide mais cette croissance vient à un coût cher. Des centaines voire des milliers de menaces contre la santé et l'environnement apparaissent chaque jour.\\

Selon l’Organisation mondiale de la santé (OMS), 3 millions de personnes meurent tous les ans à cause de la pollution de l’air. En 2015, près de 19 milliards d’euros ont été consacrés à des soins de santé lié à la qualité de l’air et 1,2 milliard de journées de travail ont été perdues. Outre les dégâts causés par la pollution de l'air, de nouvelles virus et épidémies apparaissent annuellement contre lesquels l'immunité humaine n'est pas encore développée, ce qui résulte enfin à l'apparition des maladies et des réactions allergiques parfois mortelles.\cite{1}\\

Pour lutter contre les risques présentés ci-dessus la plupart des personnes a recourt aux moyens classiques comme les masques, les gants et les cache-cols pour se protéger contre les différents agents contagieux.\\

L'inconvénient majeur de ces moyens est que les individus sont incapables de savoir au paravent le niveau de pollution ou même l'existence d'un éventuel risque à leur proximité. Et prenant en compte que la météo ne s'intéresse généralement qu'à l'état climatique, les personnes allergiques et fragiles n'ont aucun moyen pour s'informer sur l'état de l'air environnant.\\

La présence d'un moyen d'information sur la qualité quotidienne de l'air s'avère nécessaire surtout pour les personnes fragiles, asthmatiques ou atteintes par une allergie. Un tel outil indiquera le niveau de pollution de l'air en substances nocives, en bactéries et virus ou en pollen. La solution permettra aussi aux utilisateurs de s'inscrire à un type de risque pour recevoir des notifications de suivi ou de conseil. Ainsi une personne inscrite à un risque donné peut suivre les pics de pollution et recevoir en même temps des conseils de protection contre le risque. La solution est une application fiable et pérenne qui permettra de prévenir les utilisateurs de la pollution. Cette dernière fait l’objet de ce rapport où je présente la méthodologie adoptée pour répondre à sa finalité.\\

Le présent rapport décrit les étapes par lesquelles nous avons passé pour finaliser notre projet. Au début, nous présenterons le contexte général de notre application. Dans une deuxième étape, nous analyserons les besoins fonctionnels et non fonctionnels. Le troisième chapitre est consacré aux éléments de conception sur les niveaux : global et détaillé. Enfin nous avons dédié le dernier chapitre pour la réalisation dans laquelle nous présenterons les outils et l’environnement logiciel et matériel utilisés afin de réaliser notre application et nous expliquerons les interfaces de notre projet.

