\addcontentsline{toc}{chapter}{Conclusion générale}
\chapter*{Conclusion générale}

\qquad Tout au long du projet nous étions menés à concevoir et à apporter des améliorations à une application mobile permettant de suivre le niveau de pollution de l'air. L’application facilitera aux utilisateurs le suivie de l'état quotidien de l'air. Elle permet aussi à l'administrateur de suivre l'état de l'application à travers des rapports à jour sous format xls qu'il peut télécharger.\\

Ce rapport décrit toutes les phases de la réalisation du projet. Commençant par le contexte général puis la présentation de l’objectif du projet ainsi qu’une étude menée sur les solutions existantes sur le marché de l'e-santé et les limites des solutions proposées. Ensuite nous avons passé à l’analyse des besoins fonctionnels et non-fonctionnels ce qui nous a permis de fixer le cahier des charges auquel doit répondre notre application. Ce qui nous a permis d’aborder la partie de la conception où nous avons spécifié l’architecture de notre projet et ses composantes. Enfin nous avons précisé l’environnement où nous avons réalisé notre application et les outils technologiques utilisés tout au long de la phase du développement.\\

Le stage ingénieur nous a permis de consolider les connaissances acquises dans notre cursus académique, en nous donnant la possibilité d’appliquer les notions théoriques du cours qui nous ont été une base solide pour comprendre l’engin du développement mobile hybride multi-plateforme.\\

Notre application étant fonctionnelle, mais elle présente quelques imperfections. Nous projetons de lui rajouter un module qui permettra aux utilisateurs de s'interagir et de partager les informations sur les réseaux sociaux ce qui donnera plus de visibilité à l'application. Une majeure amélioration qui peut être additionnée consiste à un module de prévision des niveaux des risques à travers des algorithmes de Machine Learning et d'Intelligence Artificiel. 