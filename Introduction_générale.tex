\addcontentsline{toc}{chapter}{Introduction}
\chapter*{Introduction}

\qquad Le monde entier vie dans un état d'évolution et de croissance rapide mais cette croissance vient à un coût cher. Des centaines voire des milliers de menaces contre la santé et l'environnement apparaissent chaque jour.\\

Selon l’Organisation mondiale de la santé (OMS), 3 millions de personnes meurent tous les ans à cause de la pollution de l’air. En 2015, près de 19 milliards d’euros ont été consacrés à des soins de santé lié à la qualité de l’air et 1,2 milliard de journées de travail ont été perdues. Outre les dégâts causés par la pollution de l'air, de nouvelles virus et épidémies apparaissent annuellement contre lesquels l'immunité humaine n'est pas encore développée, ce qui résulte enfin à l'apparition des maladies et des réactions allergiques parfois mortelles.\\

Pour lutter contre les risques présentés ci-dessus la plupart des personnes a recourt aux moyens classiques comme les masques, les gants et les cache-cols pour se protéger contre les différents agents contagieux.\\

L'inconvénient majeur de ces moyens est que les individus sont incapables de savoir au paravent le niveau de pollution ou même l'existence d'un éventuel risque à leur proximité. Et prenant en compte que la météo ne s'intéresse généralement qu'à l'état climatique, les personnes allergiques et fragiles n'ont aucun moyen pour s'informer sur l'état de l'air environnant.\\

La présence d'un 
