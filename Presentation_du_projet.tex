\chapter{Présentation générale du projet}
\section*{Introduction}
\qquad Nous consacrerons le présent chapitre à la présentation du contexte général de notre Projet, commençant par l’explication de cadre général de projet et son objectif, puis nous attaquerons la partie liée à l’étude de l’existant. Finalement une clarification sur la méthodologie de travail sera entamée.

\section{Cadre du projet}

\qquad Durant notre formation d'ingénieur nous sommes menés à effectuer un stage d'été d'au moins deux mois. Le présent stage a été élaboré à Poulina Group Holding et il consiste à améliorer l'application Bware développée par Smartdata technology.

\subsection{Entreprise d'accueil} 

\qquad Le stage a été effectué au sein de Aster Training filiale du groupe Poulina.

\subsubsection{Poulina Group Holding} 

\qquad Poulina est un groupe industriel et de services tunisien. Originellement spécialisé dans l’agriculture, le groupe s’est peu à peu diversifié pour devenir le premier groupe à capitaux privés du pays.\\

Le groupe s’est organisé au fil des années autour d’une dizaine de métiers regroupant les pôles d’activités de la holding : immobilier, travaux publics, bois et biens d’équipement, produits de grande consommation, emballage, transformation de l’acier, commerce et services, matériaux de construction et intégration avicole.

\subsubsection{Aster Training}

\qquad La société Aster training est une filiale du groupe Poulina et son capital social s’élève à 145000 DT. Créée en 1996, la société Aster training assure les activités suivantes :
\begin{itemize}
	\item Organise des séminaires de formation en gestion, technique et informatique.
	\item Ingénierie informatique.
	\item Étude, conception et développements des logiciels informatiques.
	\item Étude et mise en place des réseaux informatiques.
	\item Infogérance.
	\item Audit et conseil informatiques.
\end{itemize}

\subsection{Objectif du projet}

\qquad Le but du projet est d'apporter des améliorations à l'application Bware développée par l'entreprise française Smart Data Technology. Ces amélioration toucheront la partie destinées aux clients finales, la partie serveur ainsi que la partie destinée au suivi et contrôle de l'application.

\section{Études de l'existant}

\qquad De nos jours, nous témoignons une expansion du marché des applications mobiles. Parmi ces applications on note la présence d'une catégorie destinée à la santé humaine avec un pourcentage de 1\% du marché des applications et qui voit un accroissement d'à peu près 50\% chaque année. Gardant à l'esprit la nécessité cruciale de ces applications pour certains utilisateurs que ce soit pour des raisons de suivi de leur performance physique ou leur état sanitaire (tension, glycémie, température interne ...), nous déduirons en que la catégorie santé et fitness est un marché très important dans le domaine des applications mobiles. Notre intérêt est d'étudier les applications de prévention contre la pollution. Ainsi plusieurs solutions s'offrent aux utilisateurs des mobiles garantissant le suivi du niveau de la pollution à savoir Air Visual, Plume Air Report.

\subsection{Analyse de l'existant}

\qquad Les solutions présentées ci-dessus ne couvrent pas l'intégralité des fonctionnalités nécessaires pour les utilisateurs mais elles en offrent d'autres telles que le partage de l'état de l'air dans les réseaux sociaux et la couverture des villes de partout dans le monde. Parmi les fonctionnalités absentes, nous citons :
\begin{itemize}
	\item Les notifications pour Air Visual
	\item Les détails sur les différents type de pollution pour Plume Air Report
	\item La présence de la carte
\end{itemize} 

\subsection{Critique de l'existant}

\qquad Les exemples cités précédemment ne permettent d'intégrer qu'une partie des fonctionnalités nécessaires pour le suivi du niveaux de la pollution dans des zones éparpillées dans le monde en raison d'une quarantaine de villes par pays.    