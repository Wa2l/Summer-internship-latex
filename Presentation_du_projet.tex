\chapter{Présentation générale du projet}
\section*{Introduction}
\qquad Nous consacrerons le présent chapitre à la présentation du contexte général de notre Projet, commençant par l’explication de cadre général de projet et son objectif, puis nous attaquerons la partie liée à l’étude de l’existant. Finalement une clarification sur la méthodologie de travail sera entamée.

\section{Cadre du projet}

\qquad Durant notre formation d'ingénieur nous sommes menés à effectuer un stage d'été d'au moins deux mois. Le présent stage a été élaboré à Poulina Group Holding et il consiste à améliorer l'application Bware développée par Smartdata technology.

\subsection{Entreprise d'accueil} 

\qquad Le stage a été effectué au sein de Aster Training filiale du groupe Poulina.

\subsubsection{Poulina Group Holding} 

\qquad Poulina est un groupe industriel et de services tunisien. Originellement spécialisé dans l’agriculture, le groupe s’est peu à peu diversifié pour devenir le premier groupe à capitaux privés du pays.\\

Le groupe s’est organisé au fil des années autour d’une dizaine de métiers regroupant les pôles d’activités de la holding : immobilier, travaux publics, bois et biens d’équipement, produits de grande consommation, emballage, transformation de l’acier, commerce et services, matériaux de construction et intégration avicole.

\subsubsection{Aster Training}

\qquad La société Aster training est une filiale du groupe Poulina et son capital social s’élève à 145000 DT. Créée en 1996, la société Aster training assure les activités suivantes :
\begin{itemize}
	\item Organise des séminaires de formation en gestion, technique et informatique.
	\item Ingénierie informatique.
	\item Étude, conception et développements des logiciels informatiques.
	\item Étude et mise en place des réseaux informatiques.
	\item Infogérance.
	\item Audit et conseil informatiques.
\end{itemize}

\subsection{Objectif du projet}

\qquad Le but du projet est d'apporter des améliorations à l'application Bware développée par l'entreprise française Smart Data Technology. Ces amélioration toucheront la partie destinées aux clients finales, la partie serveur ainsi que la partie destinée au suivi et contrôle de l'application.

\section{Études de l'existant}

\qquad De nos jours, nous témoignons une expansion du marché des applications mobiles. Parmi ces applications on note la présence d'une catégorie destinée à la santé humaine avec un pourcentage de 1\% du marché des applications et qui voit un accroissement d'à peu près 50\% chaque année. Gardant à l'esprit la nécessité cruciale de ces applications pour certains utilisateurs que ce soit pour des raisons de suivi de leur performance physique ou leur état sanitaire (tension, glycémie, température interne ...), nous déduirons en que la catégorie santé et fitness est un marché très important dans le domaine des applications mobiles. Notre intérêt est d'étudier les applications de prévention contre la pollution. Ainsi plusieurs solutions s'offrent aux utilisateurs des mobiles garantissant le suivi du niveau de la pollution à savoir Air Visual, Plume Air Report.

\subsection{Analyse de l'existant}

\qquad Les solutions présentées ci-dessus ne couvrent pas l'intégralité des fonctionnalités nécessaires pour les utilisateurs mais elles en offrent d'autres telles que le partage de l'état de l'air dans les réseaux sociaux et la couverture des villes de partout dans le monde. Parmi les fonctionnalités absentes, nous citons :
\begin{itemize}
	\item Les notifications pour Air Visual
	\item Les détails sur les différents type de pollution pour Plume Air Report
	\item La présence de la carte
\end{itemize} 

\subsection{Critique de l'existant}

\qquad Les exemples cités précédemment ne permettent d'intégrer qu'une partie des fonctionnalités nécessaires pour le suivi du niveaux de la pollution dans des zones éparpillées dans le monde en raison d'une quarantaine de villes par pays. Nous citons aussi d'autres fonctionnalités absentes chez les concurrents telles que les notifications, les conseils et les détails sur les différents type de pollution.

\subsection{Solution proposée}

\qquad Notre solution consiste à couvrir les déficiences des applications présentées en mettant en place une application mobile multiplate-forme destinée aux utilisateurs français qui couvrira la totalité du territoire français mais qui permettra aussi de prévenir les utilisateurs des différents risques auxquels ils sont inscrits, d'afficher des détails sur les agents nocifs et des conseils concernant les niveaux de risque élevés.\\

\section{Méthodologie de travail}

\qquad Pour mener à bien notre projet nous avons opté pour une méthodologie afin d'organiser les différentes phases du projet.\\

Alors nous avons adopté l'approche agile qui consiste à un paradigme révolutionnaire dans le monde de gestion de projet. Cet approche diffère des méthodes classiques par la migration du centre d'intérêt des étapes par lesquels passe le projet vers les étapes par lesquels passe le produit. Ainsi, cet approche élimine l'effet tunnel pour donner davantage de visibilité et de transparence en impliquant le client dans  tous les procès de la réalisation. Ce qui résulte en des majeurs changements sur tout le processus de réalisation qui devient à la fois itératif et incrémental ce qui permettra une meilleure flexibilité aux niveaux de la spécification des besoins mais en gardant un minimum de règles.\\

\subsection{Méthode Scrum}

Au sein de ce cadre méthodologique de gestion de projet, les acteurs ajustent empiriquement, au fil des itérations, leur propre méthode en fonction de leur contexte. On peut qualifier Scrum de simple, pragmatique, transparent et empirique. Scrum ne couvrant que les aspects de gestion de projet, c’est souvent la méthodeeXtreme Programming (XP) qui vient compléter le vide laissé en matière de pratiques de développement. XP apporte ainsi les pratiques de programmation en binôme, de développement piloté par les tests (TDD ou Test Driven Development), intégration continue, etc.\\
Le cycle préconisé par TDD comporte cinq étapes.
   \begin{itemize}
   	\item Écrire un premier test ;
   	\item Vérifier qu’il échoue (car le code qu’il teste n’existe pas), afin de vérifier que le test est valide ;
   	\item Écrire juste le code suffisant pour passer le test ;
   	\item Vérifier que le test passe ;
   	\item Puis ré-usiner le code, c’est-à-dire l’améliorer tout en gardant les mêmes fonctionnalités
   \end{itemize}

\section*{Conclusion}

\qquad Dans ce chapitre, nous avons présenté le cadre général de notre stage en énonçant notre entreprise d’accueil et les objectifs envisagés de ce stage. Nous nous sommes intéressés, par la suite, à l’analyse et la critique des solutions similaires déjà existantes dans le marché et à la proposition de notre solution alternative.
